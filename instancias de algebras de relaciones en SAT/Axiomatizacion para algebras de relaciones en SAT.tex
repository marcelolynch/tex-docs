\documentclass{article}
\usepackage[utf8]{inputenc}
\usepackage{ mathrsfs }
\usepackage{amsthm}
\usepackage{amsmath}
\usepackage{ dsfont }
\usepackage{amssymb}
\usepackage{titlecaps}

\usepackage{titlecaps}% http://ctan.org/pkg/titlecaps

\usepackage{fancyhdr}
\pagestyle{fancy}
\fancyhf{}
\fancyhead[R]{ITBA}
\fancyhead[L]{Marcelo Lynch}
\fancyfoot[C]{\thepage}

 \theoremstyle{definition}
\newtheorem*{fact}{Fact}
\newtheorem*{formula}{Fórmula}

\newcommand{\N}{\mathbb{N}}
\newcommand{\Z}{\mathbb{Z}}
\newcommand{\R}{\mathbb{R}}
\newcommand{\G}{\Gamma}
\renewcommand{\S}{\Sigma}

\newenvironment{modenumerate}
  {\enumerate\setupmodenumerate}
  {\endenumerate}

\newif\ifmoditem
\newcommand{\setupmodenumerate}{%
  \global\moditemfalse
  \let\origmakelabel\makelabel
  \def\moditem##1{\global\moditemtrue\def\mesymbol{##1}\item}%
  \def\makelabel##1{%
    \origmakelabel{##1\ifmoditem\rlap{\mesymbol}\fi\enspace}%
    \global\moditemfalse}%
}


\newcommand\restr[2]{{% we make the whole thing an ordinary symbol
  \left.\kern-\nulldelimiterspace % automatically resize the bar with \right
  #1 % the function
  \vphantom{\big|} % pretend it's a little taller at normal size
  \right|_{#2} % this is the delimiter
  }}


\begin{document}
\begin{center}
\subsection*{Buscando modelos de Algebra de relaciones con SAT: una axiomatización para álgebras de relaciones finitas}

\end{center}
\subsection*{Ejemplo: orden total no acotado}
\subsubsection*{ Modelado en Alloy }
Se puede modelar el hecho de que $R$ es un orden total sin un maximo en Alloy utilizando formulas que versan solo sobre la relacion $R$, la identidad y la relación universal (Top). Obviamente Alloy no encuentra un modelo pues no hay modelos finitos.

\begin{verbatim}
    sig A {
    	R: set A
    }
    
    fun Top[] : set A->A { A->A }
    fun Dom[r: set A->A] : set A->A { r.~r & iden }
    
    -- Orden total
    fact { iden in R } -- Reflexiva
    fact { R & ~R in iden } -- Antimetrica
    fact { R.R in R } -- Transitiva
    fact { ~R + R = Top[] } -- Total

    -- Predicado: "no hay maximo"
    run { no iden - Dom[R - iden] } -- No encuentra ejemplos
\end{verbatim}

\subsection*{Traduciendo las fórmulas a la lógica proposicional interpretada como algebra de relaciones}

En lo que sigue: 
\begin{itemize}
\item $At$ denota al conjunto finito de átomos. 
\item Si $i$ es un átomo y $R$ una relación, $p_{i,R}$ es una variable proposicional que se puede leer $i \in R$ 
\item Si $a$, $b$, $c$ son átomos $c_{a,b,c}$ es una variable proposicional que se puede leer $a;b \ge c$ 
\item La relación $Id$ siempre existe como la identidad (en los facts es $iden$) 
\item $Top$ es la relacion universal 
\item Podemos indicar $i \in \sim R$ con $p_{\sim i, R}$
\end{itemize}
\vspace{10px}
Las fórmulas que se presentan intentan de estar lo más cerca a lo que se podría generar automáticamente (de la forma más "naïve" posible excepto en un caso, supongo) a partir de las sentencias Alloy.
\begin{fact} $iden$ in $R$ (reflexividad)
\end{fact} 

\begin{formula} 
\[
    \bigwedge_{i\in At} (p_{i,Id} \rightarrow p_{i,R})
\]
\end{formula}

\begin{fact}
$R\ \& \sim R$ in $iden$ (antisimetría)
\end{fact}
\begin{formula}
\[
    \bigwedge_{i \in At} ((p_{i,R} \land p_{\sim i, R}) \rightarrow p_{i, Id})
\]
\end{formula}
\begin{fact}
$R.R$ in $R$ (transitividad)
\end{fact}
\begin{formula} En esta fórmula aparece la expresión para la composición relacional. Un problema es que este tipo de formulas (disyuncion de conjunciones) pueden explotar exponencialmente en tamaño al pasar a CNF. ¿Es aceptable usar traducciones equisatisfacibles en lugar de equivalentes?
\[
    \bigwedge_{i \in At} [(\underbrace{\bigvee_{j,k \in At} p_{j,R} \land p_{k,R} \land c_{jki})}_{i \in R;R } \rightarrow p_{i, R}]
\]
\end{formula}
\begin{fact}
$R + \sim R$ = $univ$ (totalidad)
\end{fact}
\begin{formula} La formula que sigue no tiene la pinta de una 'traduccion automatica' de una formula de igualdad $R1 = R2$, pero podría uno imaginarse que se puede interpretar $R = univ$ con una semantica especial ("R es la suma de todos los atomos") que resulte en algo así:
\[
    \bigwedge_{i \in At} p_{i, R} \vee p_{\sim i, R}
\]
\end{formula} 
\begin{fact}
no $iden - Dom(R - iden)$ ("no hay maximo para R": aquí $Dom(R) = R.\sim R\ \&\ iden)$
\end{fact}
\begin{formula} Esto supone que la semántica para "no R" es la conjuncion sobre todos los atomos negando su pertenencia a R.
\[
    \bigwedge_{i \in At} \lnot (p_{i,Id} \land \lnot \bigvee_{j,k \in At} [ \underbrace{p_{j,R} \land \lnot p_{j, Id}}_{j \in R - Id} \land \underbrace{p_{\sim k,R} \land \lnot p_{\sim k, Id}}_{k \in \sim (R - Id)} \land \ c_{jki} ])
\]
\end{formula}

\subsubsection*{Observaciones extras}
Hay que notar que además de estas fórmulas hace falta codificar otras formulas que le den marco de algebra relacional a lo demás. Los axiomas del álgebra booleana se cumplen gratis por la forma en la que traducimos la unión e intersección a la logica. Para el operador relacional tenemos que basta con que se cumpla:  
\begin{enumerate}
    \item $x = x;Id$   
    \item $x;y \cdot z \neq 0 \Leftrightarrow \sim x ; z \cdot y \neq 0$ 
    \item $x;y \cdot z \neq 0 \Leftrightarrow z ; \sim y \cdot x \neq 0$ 
    \item $v;x \cdot w;y \neq 0 \Leftrightarrow \sim v;w \cdot x;\sim y \neq 0$
\end{enumerate}

Para el primer axioma podemos agregar, por cada símbolo de relación $R$ que tengamos, la fórmula:
\[
    \bigwedge_{i \in At} [(\underbrace{\bigvee_{j,k \in At} p_{j,R} \land p_{k,I} \land c_{jki})}_{i \in R;Id } \leftrightarrow p_{i, R}]
\]

Los axiomas (2) a (4) son equivalentes a decir que para todos $v,w,x,y,z$ \textbf{átomos}:

\begin{modenumerate}
\setcounter{enumi}{1}
    \moditem{*} $x;y \geq z \Leftrightarrow \sim x ; z \geq y$ 
    \moditem{*} $x;y \geq z \Leftrightarrow z ; \sim y \geq x$ 
    \moditem{*} $\exists a. (v;x \geq a \land w;y \geq a) \Leftrightarrow \exists a. (\sim v;w \geq a \land x;\sim y \geq a$)
\end{modenumerate}

Veamos por ejemplo que (2) es equivalente a (2*): \\
\\
$(\Rightarrow)$ Se cumple (2) para todos los elementos del álgebra, en particular para todos los átomos. Además notemos que si $a, b, c$ son átomos:
\[ a;b \cdot c \neq 0 \Rightarrow (a;b \geq c) \vee (c \geq a;b) \Rightarrow (a;b \geq c) \vee (c = a;b)  \Rightarrow (a;b \geq c) \].

Entonces, sean $x,y,z$ átomos.
\[ 
    x;y \geq z \iff x;y \cdot z \neq 0 \iff \sim x;z \cdot y \neq 0 \iff \sim x;z \geq y
\]
\\ 
\\
$(\Leftarrow)$. Sean $X, Y, Z$ elementos cualesquiera del álgebra.
$X;Y \cdot Z \neq 0$ significa, por aditividad completa, que existen átomos $x \leq X$, $y \leq Y$, $z \leq Z$ tales que $x;y \cdot z \neq 0$.
Luego 
\[ X;Y \cdot Z \neq 0 \iff x;y\cdot z \neq 0 \iff x;y \geq z \iff \sim x ; z \geq y \iff \sim x ; z \cdot y \neq 0 \iff \sim X ; Z \cdot Y \neq 0\] 
\\
Donde la última equivalencia sale notando (de nuevo por aditividad completa) \[\sim X = \sum_{x \in At,\ x \leq X} \sim x \]
\\
\\
Estas propiedades las podemos traducir a fórmulas de la lógica proposicional en función de las variables de ciclos:

\begin{formula} $x;y \geq z \Leftrightarrow \sim x ; z \geq y$
\[ 
    c_{x,y,z} \leftrightarrow c_{\sim x, z, y}
\]
\end{formula}
\begin{formula} $x;y \geq z \Leftrightarrow z ; \sim y \geq x$
\[ 
    c_{x,y,z} \leftrightarrow c_{z,\sim y,x}
\]
\end{formula}

\begin{formula} $\exists a. (v;x \geq a \land w;y \geq a) \Leftrightarrow \exists a. (\sim v;w \geq a \land x;\sim y \geq a$)
\[
    (\bigvee_{a \in At} (c_{v,x,a} \land c_{w,y,a})) \leftrightarrow (\bigvee_{a \in At} (c_{\sim v,w,a} \land c_{x,\sim y,a}))
\] 
\end{formula}


\end{document}





